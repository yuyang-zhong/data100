\documentclass[8pt]{extarticle}
\usepackage[utf8]{inputenc}

\title{Data 100 Final Exam Reference Sheets}
\author{Yuyang Zhong}
\date{May 16, 2019}

\usepackage[margin=0.5in]{geometry}
\twocolumn

\begin{document}

\section*{Data 100 Spring 2019 Final Exam Reference}

\subsection*{Sampling}
\subsubsection*{Simple Random Sample (SRS)}
Prob. of 1 in sample of k from population of n:
$$ P = \frac{k}{n} = \frac{1}{n} + \frac{1}{n-1} + \frac{1}{n-2} + ... + \frac{1}{n-k+1}  $$
Prob. of 1 from k samples selected from n:
$$ P= \frac{{n-1\choose k-1}}{{n\choose k}} $$

\subsubsection*{Cluster Sample}
Divide population into cluster, SRS to select cluster(s)

Prob. of 1 in sample = prob of 1's cluster in sample of k cluster from pop
$$ P =\frac{\text{k clusters}}{\text{n  clusters of population}} $$


\subsubsection*{Stratified Sample}
Split into strata, SRS in \textbf{EACH} stratum

Prob of 1 in sample = prob of 1 selected from strata of n 
$$ P = \frac{1}{n} \text{, where n = strata size} $$

\subsubsection*{Multi-Stage}
Conditional event, SRS to select group, SRS to sample within selected groups
$$ P(Stage 2 | Stage 1) $$
\hline

\subsection*{Pandas}
\begin{verbatim}
#Data Frames
df.loc[row, col] #row/column names
df.loc[[name1, name2]] -> dataframe
df.loc[name1] -> series
df.iloc[row, col] #row/column indices

#Series
s.value_count() # unique value counts

grouped = df.groupby(by)
        
#Aggregate groupby object
        grouped.agg(func)
        func: max, min, fisrt, last, head
                (lambda sf: ...) 
                # manipulate subframe
                
#Filtering groupby object
        grouped.filter(func)
        func: (lambda sf: ...) 
        # filtering condition

#merge
        df.merge(df2, left_on=, right_on=)
\end{verbatim}
\hline

\subsection*{SQL}
\subsubsection*{Basic Syntax}
\begin{verbatim}
SELECT  <columns>
FROM <relations(s)/tables>
[<join type> JOIN <relation/table>] 
[ON <predicate>]
[WHERE <predicate>]
[GROUP BY <column(s)>] [HAVING <predicate>]
[ORDER BY <column(s)>] [LIMIT <number>]

ORDER BY RANDOM LIMIT # = random sample
\end{verbatim}

\subsubsection*{Joins}
Returns NULL if not found

INNER: Intersection

LEFT OUTER: keep rows on left

RIGHT OUTER: keep rows on right

WHOLE OUTER: everything


\subsubsection*{CASE Statement}
\begin{verbatim}
CASE WHEN <predicate> THEN <rv_pred> 
     ELSE <rv_else>
     END
\end{verbatim}
\hline

\subsection*{RegEx}
\subsubsection*{Characters}
\begin{verbatim}
\d: Digits [0-9]
\w: Alphanumeric [a-zA-Z0-9]
\s: White space
\D,\W,\S: Inverses of respective groups
. : any character
\ : escape to match literals
\end{verbatim}

\subsubsection*{Quantifiers}
\begin{verbatim}
* : 0 or more times
+ : 1 or more times
? : exactly 0 or 1 times
{m} : exactly m times
{m, n} : between m to n times
{m, } : at least m times
{ , n} : at most n times
\end{verbatim}

\subsubsection*{Anchors}
\begin{verbatim}
^ : beginning of line
$ : end of line
\end{verbatim}

\subsubsection*{Grouping}
\begin{verbatim}
(...) : match all characters in capturing group
[^...] : excludes all specified
\end{verbatim}

\subsubsection*{In Python}
\begin{verbatim}
import re
re.match(pattern, string) # 1 match from beginning
re.search(pattern, string) # 1 match anywhere
re.findall(pattern, string) # list of all matches
re.sub(pattern, repl, string) # replace patterns
re.split(pattern, string) #split by pattern

s.str.extract(r'[group]') # extract by group
# will only return grouped regex match
\end{verbatim}
\hline

\subsection*{Kernel Density Estimation}
Smooth function to estimate probability distribution:

Center kernel at every point

add function together and take average

$\alpha$ : standard deviation of each kernel (large: spread apart, low/flat peak; small: centered, tall peak)
\end{itemize}
\subsubsection*{Gaussian}
$$ k_\alpha(x, z)=\frac{1}{\sqrt{2\pi\alpha^2}}e^{-\frac{(x-z)^2}{2\alpha^2}} $$
\subsubsection*{Boxcar}
\[   \left\{
\begin{array}{ll}
      \frac{1}{\alpha} & -\frac{\alpha}{2} \leq x-z \leq  \frac{\alpha}{2}\\
      0 & \text{else} \\
\end{array}
\right. \]
Probability distribution area = 1

rectangle: $ \frac{1}{\alpha} \cdot \alpha $

bigger $\alpha$ more accurate

\subsection*{Dimensionality Reduction \& PCA with SVD}
Finding columns that are linearly independent in a design matrix (gives $Rank(X)$):
$$ X = U\Sigma V^\top $$
Where,

$X: n\times d$, normalized original dataset (“feature matrix”)

$U: n\times r$, left singular vectors, orthonormal

$\Sigma: r\times r$, diagonal matrix of singular values. The $k$-th singular value is the sum of the squares of the projections of the points to the line determined by the $k$-th singular vector.

$V^\top: r\times d$, right singular vectors, orthonormal\\

\textbf{Directions}

$\mathbf{v1}$ = the unit vector $\mathbf{v}$ that maximizes $|X\mathbf{v}|$

$\mathbf{v2}$ = the unit vector $\mathbf{v}$ that is orthogonal to $\mathbf{v1}$ and maximizes $|X\mathbf{v}|$

The $i$-th singular value of $X$, or $\sigma_i(X)$, is given by $|X\mathbf{v_i}|$. The first $k$ columns of $XV$ (or $UΣ$) are the first $k$ principal components\\

\textbf{Computations}

1. Centering Matrix:
\begin{verbatim}
  n = X.shape[0]
  X_transformed = (X - np.mean(X, axis=0))/np.sqrt(n)
\end{verbatim}

2. Use SVD to find PCs:
\begin{verbatim}
  u, s, vt = np.linalg.svd(X_transformed, full_matrices=False)
\end{verbatim}

3. Selecting PCs
\begin{verbatim}
  (u * s)[:,0] 
  X @ vt[0,:] 
  (X @ vt.T)[:,0]
\end{verbatim}
\hline

\subsection*{Visualizations}
\textbf{Things to Consider}

Granularity: what each row represents

Faithfulness: does data accurately capture reality

Temporality: how does data situate in date/time

Scope: coverage of data in relations to analysis (completeness) \\

\textbf{Plots}

Single Discrete: dot plot/bar plot

Single Continuous: strip plot (small), density plot/bar plot (big) 

Multiple Continuous: Scatter plot

Discrete vs. Continuous: Side-by-side box plot, overlay density plot, side-by-side violin plot

2 Discrete vs. 2 Continuous: Sub-scatter plots, color, symbols, overlays


\end{document}
